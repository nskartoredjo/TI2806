\section{Overview}
In this section we will give an overview of the developed and implemented software product. The developed product consists of a combination of software features. The two features are the Tygron Environment Connector  and the GOAL Services Virtual Human. The architecture of the product consists of the GSVH connecting to Tygron Engine, through the connector. The virtual human can retrieve information about the game and send actions to the game through the connector. The complete architecture of the product is discussed in our Architecture Design Report \cite{CTD16}. As mentioned before, the virtual human is built using the GOAL Agent Programming Language\cite{GOAL16}. This language is a Prolog based artificial intellegence language in which agents derive their choice of action from beliefs, knowledge and goals.

\subsection{Tygron Environment Connector}
Over the course of the project we had to build on the pre-existing connector which connects the GOAL bot to the Tygron game. However, it severely lacked any useful features for implementing a GOAL driven virtual human. On top of the provided connector a lot of improvements have been made over the course of the project, of which the most important are the additional actions the agent could perform, and the changes the agent could percept within the Tygron Engine. All improvements over the original connector are put in a library package (in the form of a jar file). In this package contributions of all teams related to this context are put together. This package extends the connector with more percepts (data) about the game. The extension also contains custom actions, that allow our virtual human to be able to interact in a better way with the game. 

\subsection{GOAL Service Virtual Human}
In total there are five groups that each develop their own virtual humans, so there are five different virtual humans. These take the role of municipality, a private housing corporation, the TU Delft, DUWO and services.Our virtual human fulfills the services role in the scenario. This means that our virtual human can build convenience stores, terraces and sports centers given area. The game provides the virtual human with indicators related to these three categories, which our services need to achieve. These indicators will provide us with a current value and a target value we would like to reach. The virtual human will make cognitive decisions based on these indicators and or experiences with other stakeholders present in the game, wether these other stakeholders are human or virtual humans. The virtual human is designed in such a way that it will behave like an actual human would in this game; by both reacting on other stakeholders and building believable buildings in the game.

\newpage
