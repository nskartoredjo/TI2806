\section{Introduction}

This project is built around the Tygron Engine; an urban planning 3D software product. It requires different stakeholders to work together in a simulation of a map that represents a reallife location (within a city). This allows all parties to simulate how certain decisions might influence everyone participating. All planning is done in real time, which makes the simulation more realistic. However, doing it in real time creates problems requiring every stakeholder to be physically present during the simulation. This is commonly done through Charretes \cite{Todd13}. To combat this problem we are tasked with designing and creating a virtual human, which simulates a real stakeholder by the use of artificial intelligence in such a way that it does the same actions the stakeholder would do. This virtual human should have a certain strategy and reasoning of why it acts the way it does. During simulation the virtual human will have to communicate and negotiate with other stakeholders. It is designed around a scenario created at the beginning of the project, but should be capable of acting in a multitude of scenarios in order to be of use to any actual simulation involving users of the Tygron Engine.

In this report we will reflect each aspects of the project, process and product.

First we will give an overview of the developed and implemented software product. Then, in the second chapter, a reflection is given on the product and process. This is done from a software engineering perspective. Thirdly, the developed functionalities are discussed. The fourth chapter contains a detailed explanation about our implementation of interaction design techniques. The fifth chapter describes the interaction design. In the sixth section an evaluation of all the functional modules and the product in its entirety, including a failure analysis is given. Lastly, in chapter seven, an outlook on the project is given.

