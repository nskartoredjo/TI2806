\section{Evaluation}
In this section an evaluation of the functionalities performed using a well-justified method will be presented, as well as a failure analysis – where the product does not perform as needed. This will be done for the functional modules, as well as for the product in its entirety. We will divide the functional modules into the building modules, and the modules for buying (land), selling (land) and demolishing (buildings).
\subsection{buildConvienceStore, buildSportsCenter and buildTerrace}
The first and foremost goal of our bot is to reach the indicators it is provided with, and to achieve this it needs to build buildings of a certain category in a certain area. For this we have 3 modules which work in more or less the same way: buildConvenienceStore for building convenience stores, buildSportsCenter for building sports centers and buildTerrace for building terraces. The way in which this is currently done is by first looking which has the lowest indicator at this moment, and dependent on that enter one of the three modules. Our current goal is to reach 80\% of the indicator. The building is built on a random piece of land we own. 

So as for the evaluation: currently the buildings are built, and in the right order of first building the building of which the indicator is lowest. This means the basic functionality is present. However it would be better if the location of where the building will be build is chosen at random. Instead there are a lot of factors which should influence this. For example you don't want the buildings too be to large or too small, you would not want terraces next to highways and you would not want to have two supermarkets next to each but rather spread out over different zones.

\subsection{buyLand, sellLand and demolish}
The other three functional modules we have are buyLand for buying land, sellLand for selling land and demolish for demolishing buildings. 

If the bot can't build anything and it has not met its indicators yet it will want to demolish buildings in order to have the land to build buildings it needs for its indicators. In order to do this the bot looks at all the buildings it owns but which are not buildings that contribute to one of our indicators and then demolish them. Currently this is also done in a random order. It would have been better however to also implement a strategy for this. For example first demolish buildings in zones where the bot does not yet have buildings that contribute to its indicators. 

If the bot has not met its indicators yet, does not have land without buildings on it and also does not have land with buildings on it that do not contribute to its indicators, it will try to buy land from other stakeholders. At this moment, only the basic functionality for buying land is implemented, which means land is being bought, but the land that is bought is choosen at random, there is however a maximum amount of money the bot pays for a certain area. This is however not a good way of doing this, therefor we are still working on implementing a strategy for deciding what land to buy.

Finally the bot has the sellLand module, which will only be entered once all the indicators are met. Since the bot start with very little land, the bot will need to buy more land instead of selling it. However it has achieved its indicators there is no more need for the rest of the land, therefor the bot will try to sell it for money.

\subsection{Product as a whole}
So in conclusion from the previously discussed modules: The bot can build the buildings it needs, will demolish the buildings it does not need if it needs the land to build the buildings it does need, will buy more land if it needs and sell its remaining land once the bot has reached its indicators. However it is necesary to add a strategy for deciding what buildings to build where and which land to buy, it would also be good to have a strategy for demolishing buildings but this is not strictly necessary. So we have a working bot which has all the basic functionality, however there is still room for a lot of improvement. 


\newpage
