\section{Description}
The services provider is responsible for an extend of services in the TU Delft campus area. Services include terraces, convenience stores and sports centers. These services need to be provided to all citizens of all the zones in the area. The design of our virtual human is based on the strategy discussed in \ref{sec:appendixa}. However, not every desired feature has been implemented. In this section we will discuss the features that have been developed in the virtual human.

\subsection{Indicators in the scenario}
The virtual human is able to decide the ideal locations for every service building, based on its indicators. The indicators show the amount of floor size needed per zone, based on the amount of units within the zone. This is an implementation of the real world goal for services to reach as much customers as possible. For smaller local services this means that their ideal location would be in areas with the most amount of inhabitants. Their total floor size will be based on the amount of customers they are able to reach. 

%To gather information about the amount of floor size service buildings cover in the game, we designed and implemented excel-based indicators for the game. These indicators will score each service category seperately. Every zone gets a factor appropriate to the amount of citizens of that zone. Then the floor size is compared to the amount of housing units present in the zone and if that result matches a threshold, then that zone is statified with our services. The virtual human is designed to get convenience stores, terraces and sports centers to every citizen in every zone.

\subsection{Decision making for buildings}
The virtual human is able to decide which service building it should build. It is able to decide by looking at the indicators corresponding to the service building. Looking at the indicators it tells the virtual human how much floor size is needed for a specific service building, and in which zone this building should be build. It always tries to build the largest building possible.

The virtual human will be able to buy land depending on the amount of times a service building has been build in a specific zone. This prevents the virtual human to build without the appropriate amount of available land. If the virtual human needs to build multiple buildings it could be an indicator that there isn't enough land available. This is a simplification of the desired strategies considering buying land.

If a building is build, then the virtual human is able to request a building permit from the municipality. Only if that permit is granted, the building is actually build in the game and the indicators will be updated accordingly.

The virtual human is able to demolish a building if this building is owned by the services stakeholder, but isn't a part of the services buildings. Those kind of building will be demolished, as they aren't able to contribute positively to the indicators. Demolishing those buildings also creates additional buildable land.

%By continuously consulting the indicators, the virtual human is able to decide, based on those indicators, what service building it should build. This is decided by which indicator is the lowest currently. When that decision is made, then the virtual human searches for a plot of land to build on. The build location is decided by looking at each zone and picking the zone which would allow us to provide the most citizens with our services. Next we look at the land we own ourself in that zone. If that yields a suitable place to build on, then the virtual human will build an appropriate size building on that piece of land. The size is computed based on the needed square meters of service building, that is needed in the chosen zone. If the virtual human does not already own a plot of land which suits the needs at the moment, then it will try to buy land from another stakeholder for an appropriate price. 

\subsection{Communication with other stakeholders}
The virtual human is be able to reply on requests from other stakeholders while playing the game. These might be received whenever a stakeholder wants to buy pieces of land owned by the services stakeholder. It accepts the request if the amount of money per squared meter will be above a fixed value. This prevents the virtual human to lose money considering the amount of money it could have earned while using it self.

The virtual human is also able to reply on requests from other stakeholders when they sell their land to the service stakeholder. Those requests will be accepted if the price will be below a fixed value. This to prevent spending to much money. It also take into account that the indicators might drop caused by a request to buy land. If so than those requests will be denied.

%While playing our virtual human will have to reply to request by other stakeholders in the game. These might be received whenever a stakeholder wants to buy a piece of land from the services stakeholder. The virtual human will decide if we can spare the plot of land and if we are offered a good sum of money for it. It also takes into account that, if service buildings are on that specific plot of land, our indicators might drop. Another request we might receive would be when another stakeholder offers a piece of land to us. The virtual human will decide if we want that plot of land based on the need for land and the requested sum of money.

\newpage