\section{Reflection}
In this section we will reflect on the product and the process from a software engineering perspective. We will give information about how to improve and which lessons we have learned during this project.
\subsection{Product}
The final product we have delivered, does not fulfil the expectations we had from the final product at the beginning of the project. If we could continue the project we would like to improve the strategy of the virtual human.

The developed features does consist of a solid basis, which does not require any major improvements to be made. The features could however have been extended to take more cases in account. The lack of strategy is the main missing feature that could have majorly improved the product. This could have been made possible if at the beginning we would have spend more time developing the virtual human on decision making level. This would have concluded to an early realisation that the connector needs more features to support our strategy.

The developed features in the connector, which allows us to build buildings and to buy land, could have been changed to allow more realistic behaviour. As for now, it makes choices which could be considered odd if made by a human. This behaviour includes building on roads, or building a small line of terrace. If we could redo this feature we would change it so it will take more variables in the map into account. We also like to rework the area decision making, to support the idea of having realistic behaviour.

\subsection{Process}
Our process within the group could be considered quite organized. We correctly used SCRUM to plan our sprints. We also worked together to enhance the working progress. We used a structure for pull requests to keep individual progress reviewed. Near the end of the project we added the Pull Approve applications for github, to prevent members of our group breaking the structure. For the short period of usage we favoured the addition. If we could redo this project we would add Pull Approve at the beginning of the project, including defining the rules attached to it. We could also consider a real SCRUM program, instead of using Google Spreadsheet.

We had one group member, which didn't appear on time on scheduled meetings and at a certain point didn't show up at all. The amount of work he made could be concluded as minimalistic, in time spend on the work and in actual work delivered. We learned that communicating with the TA's could help solving problems within the group. Because we wrote our hours corresponding closely to the true hours made, we were able to show that he became a problem for the progress of our project. Eventually he decided to leave the project, whereby we were left with four members. Looking back we were to forgiving in dealing with coming late and not making the required hours. If we could redo this project we would define strict rules and consequences considering the previous stated problems, to prevent piggybacking.

Overall we like the approach to meet frequently every sprint. It proves to contribute to the overall work progress. A member working at home could only work if there is enough trust, to guarantee that the member does actually work. This trust could not be given at the beginning of a project, but should be made throughout the project. If we could redo this project we would keep a strict meetings schedule, to also prevent piggybacking.