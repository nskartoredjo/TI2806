\section{Reflection}
In this section we will reflect on the product and the process from a software engineering perspective. We will give information about how to improve and which lessons we have learned during this project.
\subsection{Product}
The final product we have made, is not the product we had in mind at the start of this project. We expected to have more interaction with the bot and to have a much better strategy. At the start of this project we expected that we had to implement a bot for the Tygron game and that the connector had all the information we needed. That we had to write a strategy for our bot so it was able to play in the Tygron game. Unfortunately we were wrong. We also had to implement the connector. We had to make sure all the information we needed for the bot, was implemented in the connector. So instead of using the percept, we had to write the percepts. This caused a big change in our plan. Because instead of focusing on a strategy, we needed to focus on implementing code to get percepts before even thinking about a strategy. So our final product is now a basic bot, who is able to do basic steps, like building and buying. But a very nice strategy or interaction with other bots, is not really there. The improvement for our bot is to have this nice strategy. We weren't able to have this, because we didn't have the information implemented in the connector. Now we have. So if we had another ten weeks for this project we should be able to build the strategy we had in mind. We should be able to think of a strategy and implement this using our code from the connector, because then we don't have to implement anything new in the connector, .

\subsection{Process}
Our process was quit organized. We used SCRUM to plan our sprints and almost every day we worked together. We discussed when we needed to approve pull requests and we helped each other out when there were issues with computers. Because sometimes, some computers had problems running Tygron. Then we would run the code on another computer. We had one team member, that was always late and sometimes didn't show up at all. He also did not do much for the product. We learned that communicating with the TA's can help a lot. Because we communicated each time he didn't show up and reported the exact hours everybody spend at something, instead of just filling in so everybody had enough time, we were able to show that he was a problem in the team. But he decided to stop and we became a team of 4 members. We kept communicating with each other when somebody was late or didn't show up, so we knew what was going on. Also we kept filling in the right hours even if this caused somebody to have less than 28 hours. IF you had less, then you had to make this extra hours the next week.
An improvement of our process should be, to meet everyday and make sure the expected hours are spent. In this case we would be able to finish more tasks during a sprint.
\newpage
	