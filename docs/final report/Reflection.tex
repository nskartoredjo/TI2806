\section{Reflection}
In this section we will reflect on the product and the process from a software engineering perspective. We will give information about how to improve and which lessons we have learned during this project.
\subsection{Product}
The final product we have delivered, does not fulfill the expectations we had from the final product at the beginning of the project. If we could continue the project we would like to improve the strategy of the virtual human.

The developed features does consist of a solid basis, which major improvements does not need to be made. The features could however have been extended to take in account more cases. The lack of strategy is the main missing feature that could have majorly improved the product. This could have been made possible if at the beginning we would have spend more time developing the virtual human on decision making level. This would have concluded to an early realisation that the connector needs more features to support our strategy.

The developed features in the connector, which allows us to be able to build buildings and to buy land, could have been changed to allow more realistic behaviour. As for now, it makes choices which could be considered odd if made by a human. This behaviour includes building on roads, or building a small line of terrace. If we could redo this feature we would change it so it will take in account more variables in the map. We also like to recreate the area decision making, to support the idea of having realistic behaviour.


%The final product we have made, is not the product we had in mind at the start of this project. We expected to have more interaction with the bot and to have a much better strategy. At the start of this project we expected that we had to implement a bot for the Tygron game and that the connector had all the information we needed. That we had to write a strategy for our bot so it was able to play in the Tygron game. Unfortunately we were wrong. 

%We also had to implement the connector. We had to make sure all the information we needed for the bot, was implemented in the connector. So instead of using the percept, we had to write the percepts. This caused a big change in our plan. Because instead of focusing on a strategy, we needed to focus on implementing code to get percepts before even thinking about a strategy. 

%So our final product is now a basic bot, who is able to do basic steps, like building and buying. But a very nice strategy or interaction with other bots, is not really there. 

%The improvement for our bot is to have this nice strategy. We weren't able to have this, because we didn't have the information implemented in the connector. Now we have. So if we had another ten weeks for this project we should be able to build the strategy we had in mind. We should be able to think of a strategy and implement this using our code from the connector, because then we don't have to implement anything new in the connector.

\subsection{Process}
Our process within the group could be considered quite organized. We correctly used SCRUM to plan our sprints. We also worked together to enhance the working progress. We used a structure to for pull request to keep individual progress reviewed. We added at the end of the project the Pull Approve applications for github, to prevent members of our group breaking the structure. For the short period of usage we favoured the addition. If we could redo this project we would add Pull Approve at the beginning of the project, including defining the rules attached to it. We could also consider a real SCRUM program, instead of using Google Spreadsheet.

%Our process was quit organized. We used SCRUM to plan our sprints and almost every day we worked together. We discussed when we needed to approve pull requests and we helped each other out when there were issues with computers. Because sometimes, some computers had problems running Tygron. Then we would run the code on another computer. 

We had one group member, which didn't appear on time on scheduled meetings and at a certain point didn't show up at all. The amount of work he made could be concluded as minimalistic, in time spend on the work and in actual work delivered. We learned that communicating with the TA's could help solving problems within the group. Because we wrote our hours corresponding closely to the true ours made, we were able to show that he became a problem for the progress of our project. Eventually he decided to leave the project, whereby we were left with four members. Looking back we were to slack considering coming late, and not making the required hours. If we could redo this project we would define strict rules and consequences considering the previous stated problems, to prevent piggybacking.

Overall we like the approach to meet frequently every sprint. It proves to contribute to the overall work progress. A member working at home could only work if there is enough trust, to guarantee that the member does actually work. This trust could not be given at the beginning of a project, but should be made throughout the project. If we could redo this project we would keep a strict meetings schedule, to prevent piggybacking.

%We had one team member, that was always late and sometimes didn't show up at all. He also did not do much for the product. We learned that communicating with the TA's can help a lot. Because we communicated each time he didn't show up and reported the exact hours everybody spend at something, instead of just filling in so everybody had enough time, we were able to show that he was a problem in the team. But he decided to stop and we became a team of 4 members. 


%We kept communicating with each other when somebody was late or didn't show up, so we knew what was going on. Also we kept filling in the right hours even if this caused somebody to have less than 28 hours. IF you had less, then you had to make this extra hours the next week.

%An improvement of our process should be, to meet everyday and make sure the expected hours are spent. In this case we would be able to finish more tasks during a sprint.