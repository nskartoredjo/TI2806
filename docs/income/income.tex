\documentclass[]{article}
\usepackage{graphicx}
\usepackage[a4paper]{geometry}
\usepackage{epstopdf}
\usepackage{endnotes}

\begin{document}

	
\section{Income value}

We want our stakeholder to have an income when building a terrace, convenience store or a sport center.
In the Tygron game, you can set the value of sell price in order to simulate the money a stakeholder would get if he will build this building.
In the Tygron game, there were already default values for this, but we couldn't find where these values were based on. 
This is why we wanted to investigate what the average income was per building. \\
\\
For the Terraces we found that the sales volume per week per m2 Terrace was 104 euros. 
For convenience stores we found that the sales volume per year per m2 was 8338 euros. 
We could find any information about the average sales volumes of a sport canter, but because we want a sport center like the one that is now on the TU-Delft campus, we used the information of this building. 
The size of this building is 15000m2 and the sales volume is 7915000 euros.\\
\\
We need an sales volume per year per m2.\\
This gives use the following values:\\
For Teracces we have 5408 euros.\\
For convenience stores we have 8338 euros\\
For sport centers we have 528 euros.\\
\\
A sports center has much less income then the other buildings. This is because a sport center depents mostly on subsidies.\\
We will use these values for our stakeholder.
\\
\\
Bronnen:\\
(http://www.missethoreca.nl/hotel/nieuws/2012/4/terrasstoel-levert-per-week-172-euro-omzet-op-101120017)\\
(http://detailhandel.info/index.cfm/branches/levensmiddelenzaken/supermarkten/   )\\
(https://intranet.tudelft.nl/fileadmin/UD/ICT/TU\_Delft\_Intranet\_Besloten/Intranet\_SC/Sport\_\_\_Cultuur\_Algemeen/Organisatie/Programmabeleid/doc/Businessplan.pdf).\\
\end{document}