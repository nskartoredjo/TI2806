\section{Interactive Design}

Per lecture:

\begin{description}

\item[lecture 1:]
Since we have a simple program with very limited options, it does not seem useful to us to make a persona or sketch for a task analysis.
However, we do think it will be useful to develop scenarios that describe our system. For example to describe situation with only bots and situations with bots and humans.
We meet every week with the users so we use Contextual Inquiry to ask questions to users. However, we don’t use cultural probes since we implement our bots in a logical, structured way and don’t use the ‘artist’s inspiration’.

\item[lecture 2:]
We think GOMS are far too detailed to us for actual use, especially since we expect our users to already be familiar with Tygron and we don’t add anything extra to that interface. 
We will use cognitive walkthroughs, however due to the form of our project we will not use them at the beginning of our project, but more towards the middle/end. 
Since we don’t have any independent experts we will not use heuristic evaluation. 
Considering Empirical evaluation we will not use questionnaires Interviews or Observations, but rather experiments to see what the user will expect and how the user will react to our bots. Later we might also use field studies by using real maps we did not generate ourselves, but we will start with the TU-Campus game.


\item[lecture 3 \& 4:]
In the third and fourth lecture mostly the psychological aspects are discussed to help improve the how to of the previous mentioned interaction design methods, not so much other methods.

\end{description}

\subsection{Conclusion}
We have chosen to use scenarios, contextual inquiries, cognitive walkthroughs, and experiments.

\subsection{Meeting Plan:}
We decided to make several scenarios about how to use our product. Because you can use it with humans and our bots or with just our bots. 
We also used contextual inquiries. We went to the company Tygron every Tuesday, so we could discuss our progress and adjust to the wishes of the company. We also had regular demos for the Teacher, which is also an important stakeholder in this progress. At the end we will have an cognitive walkthrough to show how our bots work in the game.  