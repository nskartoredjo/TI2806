\section{Introduction}

Tygron is a company that builds serious games for urban planning.
A game usually consists of a scenario (such as planning a new city area) in which several players play different roles (such as major, city planner, environmental agency, etc.)
So these games are meant to be played by a group of people.
Tygron is interested in simulating some of these people with Virtual Humans, so that you can play the serious game also when not every player in the scenario is present. This is useful, because it is hard to make an appointment when everybody is available. By replacing the roles with Virtual Humans, it is easier to make an appointment and even be able to play the game on your own. In this way you can play the game more often and have more time to execute the plans for real.
\\
\\
The Virtual Human developed by us will take the role of a public service manager. Its goal is to build convenience stores, terraces and a sports center at an optimal location so that profit can be made. Our Virtual Human will play the game at the TU-Campus location with other stakeholders like DUWO and TU-Delft. Together they want to make the TU-Campus a better place to live.
\\
\\
During this project we will work using SCRUM. We will have weekly sprints and within each sprint we will implement a few features of our Virtual Human. By working in sprints, we have smaller goals and we will be able to check with Tygron more often. In this way we will be able to adapt to any changes Tygron wants us to make.
\\
\\
In this document, we will provide a high level backlog. The items mentioned in this backlog will be planned in a roadmap. Then we have a section with different user stories about how the game should work. Finally we will have a section about when the product is done.
\\
\\


\newpage