\label{product}
\section{Product}
In this section we will give a high level backlog. The items discussed in this section will be in a roadmap, to give a high level few about the planning of this project.
\subsection{High-level product backlog}

Within this section the high-level features will be defined using \textit{MoSCoW}. MoSCoW uses four categories to separate the features by level of importance. The categories are:

\begin{description}
	\item[Must Have:] Features that are of high importance. With those features the agent is considered as properly working.
	\item[Should Have:] Features that are considered favorable. Without the agent should still be functional and should properly work.
	\item[Could Have:] Features that are of low importance. When there will be enough time to implement, the features will be present.
	\item[Won't Have:] Features that won't be implemented.
\end{description}

\subsubsection{Must Have}
\begin{itemize}
	\item The stakeholder should be able to build a convenience store.
	\item The stakeholder should be able to buy land.
	\item The stakeholder should be able to sell land.
	\item The stakeholder should be able to demolish:
		\begin{itemize}
			\item The stakeholder should be able to demolish land.
			\item The stakeholder should be able to demolish buildings.
		\end{itemize}
	\item The stakeholder should be able to have a goal, such as building at least 3 convenience stores. This is called an \textit{indicator}.
\end{itemize}

\subsubsection{Should Have}
\begin{itemize}
	\item The stakeholder should be able to build more types of buildings:
	\begin{itemize}
		\item It should be able to build a terrace.
		\item It should be able to build a sports center.
	\end{itemize}
	\item The stakeholder should have a low Level of communicating:
	\begin{itemize}
		\item It should have information about the changing surroundings.
		\item It should be able to calculate an efficient placing of a building.
	\end{itemize}
	\item The stakeholder should be able to achieve its goals:
	\begin{itemize}
		\item It shouldn't go bankrupt.
		\item It should reach its indicators considering convenience stores.
		\item It should reach its indicators considering terraces.
		\item It should reach its indicators considering the sports center.
	\end{itemize}
\end{itemize}

\subsubsection{Could Have}
\begin{itemize}
	\item The stakeholder could be able to understand the used language between stakeholders.
	\item The stakeholder could be able to negotiate with other stakeholders:
	\begin{itemize}
		\item It could be able to barter prices.
	\end{itemize}
\end{itemize}

\subsubsection{Won’t Have}
\begin{itemize}
	\item The stakeholder won't interact with human stakeholders:
	\begin{itemize}
		\item It won't understand messages send by human stakeholders.
		\item It won't send messages to humans.
	\end{itemize}
\end{itemize}

\subsection{Roadmap}
In this section we describe a high level planning for the items stated in the section above. Per sprint we select a part of the product we are going to develop during this sprint. \newline

\subsubsection{Sprint 1}
During this sprint, we will find out how the software works and make a start with the product vision, the product plan and the architectural design.
\begin{itemize}
	\item Try different scenarios and stakeholders when using the software of Tygron.
	\item Make the product vision document.
	\item Make a product plan.
	\item Think of a scenario with different stakeholders and a conflict.
\end{itemize}

\subsubsection{Sprint 2}
During this sprint we will find information about our stakeholder and make a final scenario. Also we will brainstorm about how to implement the basic action for our stakeholder.
\begin{itemize}
	\item We should choose an indicator for our stakeholder.
	\item We have to make a demo of the final scenario.
	\item We need to find out how we can make the stakeholder be able to build a convenience store.
	\item We need to find out how we can make the stakeholder be able to  demolish land and buildings.
\end{itemize}

\subsubsection{Sprint 3}
During this sprint we will make sure the stakeholder is able to do all the basic actions during the game.
\begin{itemize}
	\item The stakeholder should be able to buy and sell property.
	\item The stakeholder should be able to demolish buildings and land.
	\item The stakeholder should be able to build in context to its indicators.
\end{itemize}

\subsubsection{Sprint 4}
During this sprint our stakeholder should be able to reach his goals.
\begin{itemize}
	\item The stakeholder should be able to reach its indicators target.
	\item The stakeholder should be able to make sure it doesn't go bankrupt.
	\item We need to make a demo in order to show how our stakeholder is able to reach its goal.
\end{itemize} 

\subsubsection{Sprint 5}
During this sprint the stakeholder should be able to notice what other stakeholders do.
\begin{itemize}
	\item The stakeholder should be able to notice the other stakeholders.
	\item The stakeholder should be able to keep track of the decision of other stakeholders.
\end{itemize}

\subsubsection{Sprint 6}
During this sprint the stakeholder should be able to calculate efficient placing of buildings.
\begin{itemize}
	\item The stakeholder should be able to calculate if owned property is good for building.
	\item The stakeholder should be able to calculate if other land is good for building.
\end{itemize}

\subsubsection{Sprint 7}
During this sprint we could add more indicators to the stakeholder.
\begin{itemize}
	\item The stakeholder is able build more shops
	\item The stakeholder is able to build parking lots.
	\item We need to make a demo in order to show the finished stakeholder
\end{itemize}

\subsubsection{Sprint 8}
During this sprint we will end everything about the stakeholder and make sure it still works with the other stakeholders made by other groups.
\begin{itemize}
	\item run final tests with other teams.
	\item clean all code.
	\item deliver the code.
	\item deliver all documentation.
\end{itemize}
\newpage
