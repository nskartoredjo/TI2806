\label{product}
\section{Product}

\subsection{High-level product backlog}

Within this section the high-level features will be defined using \textit{MoSCoW}. MoSCoW uses four categories to separate the features by level of importance. The categories are:

\begin{description}
	\item[Must Have:] Features that are of high importance. With those features the agent is considered as properly working.
	\item[Should Have:] Features that are considered favourable. Without the agent should still be functional and should properly work.
	\item[Could Have:] Features that are of low importance. When there will be enough time to implement, the features will be present.
	\item[Won't Have:] Features that won't be implemented.
\end{description}

\subsubsection{Must Have}
\begin{itemize}
	\item Build one type of building (non contextual)
	\item Buy land
	\item Sell land
	\item Demolish:
		\begin{itemize}
			\item Land
			\item Buildings
		\end{itemize}
	\item Own indicator(s)
\end{itemize}

\subsubsection{Should Have}
\begin{itemize}
	\item Build more types of buildings:
	\begin{itemize}
		\item shops
		\item parking lots
		\item Horeca
	\end{itemize}
	\item Low Level Communicating:
	\begin{itemize}
		\item Information about the changing surroundings
		\item Calculate efficient placing of buildings
	\end{itemize}
	\item Achieve Goals:
	\begin{itemize}
		\item Don’t go bankrupt
		\item Reach indicators considering shops
		\item Reach indicators considering parking lots
		\item Reach indicators considering horeca
	\end{itemize}
\end{itemize}

\subsubsection{Could Have}
\begin{itemize}
	\item Understanding the used language between stakeholders
	\item Negotiating with other stakeholders:
	\begin{itemize}
		\item Be able to barter prices
	\end{itemize}
\end{itemize}

\subsubsection{Won’t Have}
\begin{itemize}
	\item understanding messages send by human stakeholders.
	\item interacting with human stakeholders:
	\begin{itemize}
		\item understanding messages send by human stakeholders.
		\item sending messages to humans
	\end{itemize}
\end{itemize}

\subsection{Roadmap}
Below we described per sprint which part of the product we are going to make. \newline

\subsubsection{sprint 1}
During this sprint, we will find out how the software works and make a start with all documentation.
\begin{itemize}
	\item Try different scenarios and stakeholders when using the software of Tygron.
	\item Make the product vision document.
	\item Make a product plan.
	\item Think of a scenario with different stakeholders and a conflict.
\end{itemize}

\subsubsection{sprint 2}
During this sprint we will make a final scenario and find information about our stakeholder. And make sure the stakeholder can do some of the basic actions.
\begin{itemize}
	\item Choose a indicator for our stakeholder.
	\item Make demo of basic implementation of stakeholder.
	\item The stakeholder should be able to build one type of building.
	\item the stakeholder should be able to demolish land and buidlings.
\end{itemize}

\subsubsection{sprint 3}
During this sprint we will make sure the stakeholder is able to do all the basic actions during the game.
\begin{itemize}
	\item The stakeholder should be able to buy and sell property.
	\item The stakeholder should be able to build in context to its indicator.
\end{itemize}

\subsubsection{sprint 4}
Durig this sprint our stakeholder should be able to reach his goals.
\begin{itemize}
	\item The stakeholder should be able to reach its indicators target.
	\item The stakeholder should be able to make sure it doesn't go bankrupt.
	\item Make demo in order to show how our stakeholder is able to reach its goal.
\end{itemize} 

\subsubsection{sprint 5}
During this sprint the stakeholder should be able to notice what other stakeholders do.
\begin{itemize}
	\item Make the stakeholder notice the other stakeholders.
	\item Keep track of the decision of other stakeholders.
\end{itemize}

\subsubsection{sprint 6}
During this sprint the stakeholder should be able to calculate efficient placing of buildings.
\begin{itemize}
	\item The stakeholder should be able to calculate if owned property is good for building.
	\item The stakeholder should be able to calculate if other land is good for building.
\end{itemize}

\subsubsection{sprint 7}
During this sprint we could add more indicators to the stakeholder.
\begin{itemize}
	\item The stakeholder is able build more shops
	\item The stakeholder is able build horeca
	\item The stakeholder is able to build parking lots.
	\item Make Demo in order to show the finished stakeholder
\end{itemize}

\subsubsection{sprint 8}
During this sprint we will end everything about the stakeholder and make sure it can work with the other stakeholders made by other groups.
\begin{itemize}
	\item run tests with other teams.
	\item clean all code.
	\item deliver the code.
	\item deliver all documentation.
\end{itemize}
\newpage
