\section{Comparable products}
In this section all comparable products in the field are layed out next to our product. Comparable products are urban planning software and implementations of virtual humans.  \newline

\subsection{Own product}
How does our product compare to Tygrons own product?
The Tygron Engine currently is only playable through human interaction. There are at the time no available programs that simulate real life stakeholders, virtual humans. \newline

\subsection{Competitive products}
How does our product compare to competitors of Tygron?
Urban planning software used in current day scenarios are not based on the ability to collaborate and compete with different stakeholders in the same session. Virtual humans, who actually plan urban spaces following specified casi, are not yet commercially available. \newline

\subsubsection{Urban planning software}
Over the years there has been a substantial amount of games related to urban planning\cite{Poplin11}, but only recently serious games have been up and coming. The Environmental Systems Research Institute is the developer of the Geographic Information System\cite{GIS06}. Based on this system, a lot of applications are build which allow users to plan urban environments and share these plans with other urban planners. An example of such an extension is iCity\cite{Steve07}. This tool allows users to import geodata and simulate a 3d world. The big difference with the Tygron Engine is that the focus is more on modelling than on simulating conflicts between different stakeholders. \newline

\subsubsection{Virtual human implementations}


\subsection{Unique selling points}
Our product allows customers of the Tygron Engine to simulate urban planning scenarios on a more regular basis. Customers are not bound by strict schedules of all different stakeholders and are able to replace certain stakeholders if need be. \newline
