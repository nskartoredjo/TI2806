Per college:

1
Aangezien wij een vrij simpl programma met gelimiteerde opties hebben, lijkt het ons niet nodig een persona te maken of een taak analyse te schetsen. 
Echter vinden wij het wel nuttig om scenario’s te ontwikkelen om ons systeem te beschrijven.
We meet every week with the users so we use Contextual Inquiry to ask questions to users. However, we don’t use culturual probes since we implement our bots in a logical, structured way and don’t use the ‘artist’s inspiration’.

2
We think GOMS are far too detailed to us for actual use, especially since we expect our users to already be familiar with Tygron and we don’t add anything extra to that interface. 
We will use cognitive walkthroughs, however due to the form of our project we will not use them at the beginning of our project, but more towards the middle/end. 
Since we don’t have any independent experts we will not use heuristic evaluation. 
Considering Emperical evaluation we will not use questionaires Interviews or Observations, but rather experiments to see how the user will epect and react to our bots and field studies by using real maps we did not generate ourselves.


3
Hier wordt vooral de menselijk psychologie besproken en niet zozeer meer methodes voor HCI, echter wel dingen die daarbij komen kijken en waar rekening mee zal worden gehouden wat echter niet specifiek vermeld word.

4.
Ook meer dingen over hoe, maar geen nieuwe methodes.